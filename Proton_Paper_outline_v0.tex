\documentclass{article} 
\usepackage[margin=1.0in]{geometry}
\usepackage{outlines}
\usepackage{lineno}
\usepackage{color}
\usepackage{amsmath,accents}
\usepackage{mathrsfs}

\linenumbers
\usepackage{enumitem}
\setlistdepth{8}

\newlist{myEnumerate}{enumerate}{9}
\setlist[myEnumerate,1]{label=(\Roman*)}
\setlist[myEnumerate,2]{label=(\Alph*)}
\setlist[myEnumerate,3]{label=(\roman*)}
\setlist[myEnumerate,4]{label=(\alph*)}
\setlist[myEnumerate,5]{label=(\arabic*)}
\setlist[myEnumerate,6]{label=(\Roman*)}
\setlist[myEnumerate,7]{label=(\Alph*)}
\setlist[myEnumerate,8]{label=(\roman*)}

\begin{document} 
\author{David M. Green}
\title{Fermi-LAT Measurement of Cosmic-ray Proton Spectrum \\ Paper Outline - Version 0}
\date{\today}

\maketitle

\begin{abstract}
The Pass 8 gamma-ray simulation and reconstruction package for the Large Area Telescope (LAT) on the Fermi Gamma-ray Space Telescope has allowed for the development of a new cosmic-ray proton analysis. 
Using the Pass 8 direction and energy reconstruction, we create a new proton event selection. 
This event selection has an acceptance of 1 m$^2$ sr over the incident proton energy range from 50 GeV to over 8 TeV and when applied to over 7 years of LAT observations provides over 700 million events for a spectral measurement. 
The systematic errors in the acceptance and energy reconstruction require careful study and will contribute significantly to the spectral measurement. 
The event selection and spectral measurement of the Pass 8 proton analysis opens the door to additional proton analyses with the LAT, such as the evaluation of proton anisotropy. 
We present a detailed study on the measurement of the cosmic-ray proton spectrum with Pass 8 data for the Fermi LAT.
\end{abstract}

\section{Introduction}
	\begin{myEnumerate}
		\item Describe overview of LAT
		\begin{myEnumerate}
			\item Launch date to give early context of how much data there is available
			\item Orbital parameters to show what kind of space environment we have to deal with
			\item Development of Pass 8, short list of improvements and how this enables use to make a new proton analysis with the LAT
		\end{myEnumerate}
		\item Discuss recent developments of the CR proton spectrum from instruments
		\begin{myEnumerate}
			\item AMS-02 observes break in spectrum at 300 GeV
			\item Potentially resolves discrepancy between satellite measurements in 100s GeV energy and balloon-borne measurements 
			\item But.... AMS-02 only goes to 1.8 TeV, statistics limited due to small acceptance and X years of flight
			\item Gap left between 1.8 TeV of AMS-02 and 3 TeV of CREAM
		\end{myEnumerate}
		\item Goals of this analysis
		\begin{myEnumerate}
			\item Measure the cosmic-ray proton spectrum from 50ish GeV to several TeV
			\item Fermi LAT in unique position to measure spectrum spanning between satellite measurements and balloon borne measurements
			\item Also able to confirm spectral break as currently only seen by AMS-02 and possibly by Pamela
			\item Create a new data set of cosmic-ray protons for future analysis \textcolor{red}{(I'm not sure we really need this in the paper but might be nice to mention)}
		\end{myEnumerate}
		\item Event selection for high quality proton sample
		\item Energy reconstruction, biases, energy resolution, and limitations
		\item Describe out instrument response: acceptance and contamination
		\item Describe the methods used for spectral reconstruction: unfolding and forward folding using response matrix derived from MCs
		\item Describe evaluation of systematic uncertainties
		\begin{myEnumerate}
			\item Due to event selection: acceptance and contamination
			\item Energy measurement: absolute energy scale and energy resolution
			\item From hadronic model of Geant4 simulations
			\item Spectral reconstruction: comparing unfolding and forward folding methods
		\end{myEnumerate}
		\item Finally discuss observations and features of measured spectral, including possible spectral break and agreement with recent results \textcolor{red}{(definitely need to but this in context with other measurements since while energy resolution is poor and systematics less precises than AMS-02 we can extend the energy further into the region of balloon-borne detectors which have never been done before and makes a quantitative connection between two different observation environments)}
	\end{myEnumerate}

\section{Event Analysis}
	\begin{myEnumerate}
		\item Overview
			\begin{myEnumerate}
				\item Description of the LAT
				\item 4$\times$4 array of towers which measure direction and energy of incoming cosmic-ray
				\item Each tower is composed of TKR and CAL
				\item TKR information
				\begin{myEnumerate}
					\item Each TKR module is 18 x-y planes of silicon-strip detectors with tungsten converter foil
					\item Total of 1.5 $X_0$ at normal incidence (should convert this to nuclear interaction length)
					\item X-Y nature and depth of TKR allows for determination of initial direction of cosmic-ray
					\item Additionally able to measure the time over threshold of CR
					\item ToT allows for measurement of signal $\propto$ Z$^2$
					\item The last 4(?) Tungsten converter foils are thicker than the previous layers to ensure gamma-rays convert within TKR
				\end{myEnumerate}
				\item CAL information
				\begin{myEnumerate}
					\item CAL is homogeneous electromagnetic calorimeter
					\item Each CAL module is 96 CsI(Tl) crystals in an hodoscopic array in 8 layers.
					\item The hodoscopic nature of the CAL allows for measuring the shape and evolution of each particle shower which can be used with a profile fitter to determine the incident energy of the cosmic-ray
					\item additional the imagine capability of the CAL allows for the measurement of the direction of the incident CR
					\item At normal incidence the CAL is 0.5 $\lambda_i$ lengths deep but at horizontal incidence is it 1.5 $\lambda_i$ deep
				\end{myEnumerate}
				\item Anti-coincidence detector (ACD) surrounds the 4$\times$4 tower array
				\item ACD information
				\begin{myEnumerate}
					\item 89 segmented covering 5 sides of the tower array
					\item Each tile independently measures deposited energy from CR
					\item Deposited energy $\propto$ Z$^2$ 
				\end{myEnumerate}
				\item Description of the LAT triggers and filters and point towards the paper with more information
				\item LAT was not designed for accurate measurement of hadronic showers
				\begin{myEnumerate}
					\item Very shallow homogeneous calorimeter not idea for fully capturing energy hadronic shower profile
					\item Compare to CREAM and/or AMS-02 
					\item Unable to measure energy on an event by event basis, need to focus on a statistical ensemble approach with high event rate
					\item Therefore need to be aware of limitation of energy measurement and associated systematic uncertainties
				\end{myEnumerate}
			\end{myEnumerate}
		\item Pass 8 Event Reconstruction 
			\begin{myEnumerate}
				\item \textcolor{red}{I'm not 100\% sure of the depth of this section but seeing as though we are using Pass 8 and that was a somewhat critical step into enabling this analysis's possibility I think having a dedicated section in the Event analysis chapter might make sense.  If we put it anywhere it should be rather early before the simulations and after the describing the instrument}
				\item Pass 8 is the new event reconstruction and simulation software developed by the Fermi LAT collaboration that drastically improves LAT's performance
				\item New event classification using boosted decision trees in TMVA
				\begin{myEnumerate}
					\item Several new Pass 8 variables have been created to determine the quality of the direction, energy, and gamma-ray quality
				\end{myEnumerate}
				\item More variables gives better separation between hadronic and leptonic showers in TMVA
				\item Improved profile to fitting to particle showers improves energy measurement
				\begin{myEnumerate}
					\item The New Full Profile fitter is able to extend the longitudinal profile of the shower outside the CAL therefore estimating the amount of energy leakage for high energy events, $>$ 100 GeV
					\item Two energies derived from new full profile fitter, one for TKR directions and one for CAL directions
				\end{myEnumerate}
				\item New tree based TKR reconstruction allows for direction reconstruction at higher angles and larger energies
				\item New ACD reconstruction provides better particle identification, lowering the contamination of proton sample
				\item \textcolor{red}{Is there something else I am missing from Pass 8?  There is no Pass 8 paper to reference this so I am not sure how in depth I should go into this discussion.}
			\end{myEnumerate}
		\item Monte Carlo simulations
			\begin{myEnumerate}
				\item Need to stress the importance of the simulations since this is how we derive all of our instrument response functions 
				\item Also use simulations for the development of TMVA selection to remove contamination for other CRs
				\item Simulations based on Geant4
				\item LAT instrument and spacecraft are fully simulated within Geant4
				\item Particles with distributions of energies, directions, and charges are generated and propagated with realistic physics models for interactions with the simulated LAT which create raw data
				\item Raw simulated data is processed through the same Pass 8 reconstruction software as flight data
				\item We preform extensive comparison between simulated data and flight data to ensure results from MC analyses can be reliably applied flight data
				\item Three types of simulations are used this analysis:
				\item Proton simulation
				\begin{myEnumerate}
					\item Simulation run from 4 GeV to 20 TeV
					\item Cover 4$\pi$ sr 
					\item Created with an $dN/dE \propto E^{-1.5}$ spectral index
					\item Original purpose to study Pass 8 CR rejection for studying extragalactic background light 
					\item This produces a simulation event sample of over X million events
				\end{myEnumerate}
				\item Electron simulation
				\begin{myEnumerate}
					\item  10 GeV to 10 TeV
					\item Cover 2$\pi$ sr (the top half of the instrument)
					\item Created with a $dN/dE \propto E^{-1.0}$ spectral index
					\item Original purpose of studying instrument response for cosmic-ray electron analysis
					\item This produces a simulation event sample of over X million events
				\end{myEnumerate}
				\item Background simulation 
				\begin{myEnumerate}
					\item The background simulation was created to accurately simulate the cosmic-ray environment of the LAT during space flight
					\item It contains CR particle from Z = 1 to Z = 26, electrons, positrons, neutrons, and Earth albedo gamma-rays
					\item All particles are simulated with realistic fluxes using results from recent CR experiments
					\item The background simulation used in this analysis simulates about 8 days worth of livetime 
					\item Protons range: 4 GeV - 10 TeV and 4$\pi$ sr
					\item Electrons/Positrons range: 4 GeV - 10 TeV and 4$\pi$ sr
					\item Helium range: 4 GeV - 20 TeV and 4$\pi$ sr
					\item Heavier CR range: 2 GeV/amu - 50 GeV/amu and 4$\pi$ sr
					\item Fluxes are taken to be near solar minimum
				\end{myEnumerate}
				\item All simulations are produces with an additional setting called overlay events
				\begin{myEnumerate}
					\item Overlay events are created from diagnostic events from flight data and signal is added on top of the simulated data
					\item This is mimic the effect of having two events simultaneous enter the LAT (for such high events rates at lower energies is a reasonable assumption)
					\item Pass 8 has many new algorithms to handle and reduce the effect of two simultaneous events interacting with the LAT
				\end{myEnumerate}

			\end{myEnumerate}
		\item Event Selection
			\begin{myEnumerate}
				\item Minimum Quality Cuts
				\begin{myEnumerate}
					\item We want a selection of hight quality protons for the spectral analysis
					\item TkrNumTracks $>$ 0: Require an event to have at least one track
					\item WP8CTPSFTail $>$ 0.5: WP8CTPSFTail is a new Pass 8 TMVA variable which determines the quality of direction reconstruction, this ensures the track is well reconstructed
					\item CalEnergyRaw $>$ 20 GeV: This utilizes the high pass filter on the LAT.  Any event with a deposited energy greater than 20 GeV is downloaded from the LAT.  This means for events with CalEnergyRaw $>$ 20 GeV is not effected by and lower energy filters which are difficult to understand for protons
					\item CalTrackAngle $<$ 0.3: Ensure the difference between the CAL and TKR directions is small.
					\item CalNewCfpSat: Ensure that the variables resulting new full profile fitter are not saturated
					\item Tkr1LengthInCal $>$ 200.0: Want a long path length through the calorimeter and does not fall within gaps of CAL.  More active material will help ensure more deposited energy and a better reconstructed energy
					\item CalLeakCorr $>$ 0.25: CalLeakCorr is a correction for the amount of shower leakage out of the CAL.  This is defined for electromagnetic showers and therefore poorly represents the intrinsically wider and more stochastic nature of hadronic showers.  We want to minimize the effect of this variable on the analysis.  
					\item log10(TkrTree1ThickRLnNodes) $<$ 1.0: To make sure events convert within the beginning of CAL and not last few layers of TKR (don't want to lose energy to TKR) we ensure the last few layers of the TKR do not have too many events  This also helps with backslash and again losing energy back into the TKR
					\item These cuts ensure the direction and energy are well reconstructed, still need remove contamination source from helium + heavier CRs and electrons
				\end{myEnumerate}
				\item Helium and Heavier Ion Cut
				\begin{myEnumerate}
					\item To remove $|Z| > 1$ CRs we use two independent measures of charge in the LAT
					\item Tkr1TotTrAve
					\begin{myEnumerate}
						\item The average time over threshold of signal in TKR
						\item Signal is ionization of CR interacting with silicon trips, therefore Signal $\propto Z^2$ in units of MIPs
					\end{myEnumerate}
					\item Acd2PLCTkr1TileActDistEnergy
					\begin{myEnumerate}
						\item The path length corrected energy deposited in the ACD in units of MeV
						\item Signal once again due to ionization of CR interacting with plastic scintillator, therefore AcdPlcEnergy $\propto Z^2$
					\end{myEnumerate}
					\item Using these two measures of charge we create of phase-space of Tkr1TotTrAve vs  \\Acd2PLCTkr1TileActDistEnergy 
					\item Designate a polygon in that phase-space around Z$^2$ = 1 using BKG simulation
					\begin{myEnumerate}
						\item It should be noted that Geant4 does not accurately recreate the rate of heavy ions interacting with the LAT, it greatly under estimates the event rate
						\item But since ionization is relatively simple and these heavy CRs aren't showering in the ACD or TKR, the amount of ionization energy is fairly accurate
						\item What this means is we can define the polygon and trust the position of the polygon in flight data
					\end{myEnumerate}
					\item We also include an additional cut on Acd2Cal1TriggerEnergy15
					\begin{myEnumerate}
						\item Acd2Cal1TriggerEnergy15 is the energy deposited around a $15^{\circ}$ cone from the CAL direction
						\item The reason for this is a population of large angle heavy CRs priests after the cut, they tend to have a poor TKR direction reconstruction but good CAL direction reconstruction.  We demand the Acd2Cal1TriggerEnergy15 < 30 MeV
					\end{myEnumerate}
					\item This greatly reduces the contamination of heavier ions and alphas to under $1\%$ and decreasing at higher energies
					\item Also since ionization is $\propto Z^2$ we cannot differentiate protons from electrons using this cut
				\end{myEnumerate}
				\item Proton Classifier
				\begin{myEnumerate}
					\item To remove electrons from the proton sample we develop an event classifier with TMVA using protons as signal and electrons and background
					\item Since leptonic and hadronic showers are fundamentally different we can use several variables of merit from Pass 8 reconstruction to build a multi-variate analysis and differential protons from electrons
					\begin{myEnumerate}
						\item The CALs ability to image showers means we have several variables with trace shower size and evolution (hadronic showers are much wider and longer than leptonic showers)
						\item Additional variables like centroid position of the shower, and size of the track left in the TKR also are able to distinguish between leptonic and hadronic showers
					\end{myEnumerate}
					\item Using simulations previously described with a boosted decision tree with TMVA to train classifier
					\item Using classifier output we can select on an optimal value for each energy bin and create an energy dependent event selection
					\begin{myEnumerate}
						\item This is critical because the proton classifier's ability to distinguish protons from electrons is energy dependent
						\item Classifier has trouble separating events at high energies (all events look very similar at highest energies since instrument is small compared to size of TeV showers)
					\end{myEnumerate}
					\item The optimal value for each energy bin is selected by using the ROC curve (signal efficiency to background rejection) ensuring the derivative of the ROC curve is below a defined tolerance
					\item We also use several different proton scanning efficiencies
					\begin{myEnumerate}
						\item Find energy dependent cut on proton classifier output which ensures a predefined (90$\%$,80$\%$, 70$\%$, etc...) is achieved
						\item This allows us to probe systematic errors associated with the proton classifier, acceptance, and contamination
						\item We produce a spectrum for each different scanning efficiency and compare it to the optimized selection
					\end{myEnumerate}
					\item We use a template fitter developed by the Fermi-LAT electron analysis to evaluate the data/MC agreement for the variables used in the training the classifier and the output of the classifier
					\begin{myEnumerate}
						\item Data/MC agreement is very important
						\item MC needs to properly recreate data variables in order trust classifier is giving the desired results
						\item Discrepancies between Data/MC in training variables will translate to systematic errors associated with classifier
						\item Also dependent on how important the variable is for training the classifier
						\item Data/MC agreement for classifier output is very good
						\item Hopefully systematic uncertainties are low
					\end{myEnumerate}
				\end{myEnumerate}
			\end{myEnumerate}
		\item Energy Measurement
			\begin{myEnumerate}
				\item Using the profile fitter developed for Pass 8 gamma-ray reconstruction we can estimate the energy of the incident proton
				\begin{myEnumerate}
					\item The shallow geometry of the CAL mean the shower is not completely contained within CAL and there is significant leakage outside the CAL
					\item The profile fitter can help mitigate this effect 
					\item The profile fitter has some large limitations, it was designed for electromagnetic showers
					\item Therefore it only has one length component instead of the two component profile usually used for hadronic showers
					\item This translates to the reconstructed energy typically has large energy dependent bias and rather poor energy resolution
					\item We specifically use CalNewCfpCalEnergy for our reconstructed energy variable.
					\item This variable uses the CAL direction instead of the TKR direction and does not assume the proton begins showering with in TKR, therefore better accounting for the start of the shower and reconstructing the energy a bit better at energies above 100 GeV
					\item Still better than using CalEnergyRaw
				\end{myEnumerate}
				\item Energy resolution and bias
				\begin{myEnumerate}
					\item Because of the limitations of the profile fitter it can be difficult to measure the energy resolution
					\item To do so we use Proton simulations and find $E_{rec}/E_{true}$
					\item We fit this distribution with function DEFINE FUCNTION LATER and find the with and bias of the distribution
					\item Because of the side distribution we cannot simply add missing energy, therefore we need to scale with distribution in order unbias the distribution
					\item From this unbiased distribution we can find the energy resolution
					\item Energy resolution tends to increase at as energy increases (not unexpected since at higher energies the shower will leak more and more out of the CAL)
					\item $68\%$ energy resolution is stable between 100 GeV and 3 TeV at about $\sim 30\%$
					\item Definitely not the best but purpose of study is extend energy measurement no to have the best energy resolution
					\item This translates to about 12 energy bins in the final spectrum
				\end{myEnumerate}
				\item Minimum Energy
				\begin{myEnumerate}
					\item The minimum energy is determined from the HiPass filter
					\item The HiPass filter begins at CalEnergyRaw $>$ 20 GeV
					\item Since CalEnergyRaw has such a poor energy resolution (compared to CalNewCfpCalEnergy) this sets up a minimum energy that can be used for the spectral analysis
					\item The mean of this distribution is 54 GeV.  This means the minimum energy we can reliably expect is 54 GeV
				\end{myEnumerate}
			\end{myEnumerate}
	\end{myEnumerate}

\section{Spectral Analysis}
	\begin{myEnumerate}
		\item Instrument Acceptance
		\begin{myEnumerate}
			\item Acceptance is defined as the produce to the instrument field of view and the effective area
			\item It is calculated using the proton MC across the entire desired energy range and angular range in true energy instead of reconstructed energy
			\begin{equation}
			\mathscr{A_{eff}}(E) = 4\pi sr \times A_{gen} \times \frac{N_\text{pass}(E)}{N_\text{gen}(E)} 
			\end{equation}
			\begin{myEnumerate}
				\item Effective area is simply calculated by dividing the number of events passing the event selection in each energy bin and the total number of events generated in that energy bin and then multiply by the generation area (typically $6m^2$)
				\item The energy variable is done in true energy
			\end{myEnumerate}
			\item We find acceptance for the optimal event selection and each constant signal scanning efficiency
			\begin{myEnumerate}
				\item The optimal event selection has a maximum acceptance of about 0.2 m$^2$ sr
				\item We ideally want flat acceptance across as most of the energy range as possible
				\item This is to ensure potential spectral features are not from changes in the acceptance
				\item Fall of off acceptance above a few TeV is due events not passing selection cut and large saturating effects and small instrument not being able to contain entire shower
			\end{myEnumerate}
			\item We use a large enough simulation such that statistical error associated with acceptance is $<1\%$
		\end{myEnumerate}
		\item Residual Contamination
		\begin{myEnumerate}
			\item Two sources of contamination: Z $>$ 1 CRs and Electrons
			\item Heavier CR Contamination
			\begin{myEnumerate}
				\item Helium is the primary sources in the contamination source
				\item The rest of the heavier CRs typically have a harder spectrum and are easier to remove than Helium (the $Z^2$ dependence more easily separates higher Z elements)
				\item To estimate the residual contamination we use BKG simulation and scale the event rates for helium and heavy CRs to flight rates using the template fitter and find an overall correction factors
				\item We then use those correction factors for helium and heavier CR for event rates after applying the cut (we have to do this because after applying the cut there are not enough event rates to determine the correction factors)
				\item This results in the helium cut effective reducing the contamination to less than $1\%$ (predominately at $\sim$50 of GeV)
				\item We know this is not an effect of no events at higher energies from generation because the maximum generation energy for helium is 20 TeV and 50 GeV/amu for other elements (which often is about to 1TeV in total kinetic energy)
			\end{myEnumerate}
			\item Electron Contamination
			\begin{myEnumerate}
				\item The proton classifier is the main method of reducing electron contamination
				\item Electron contamination before the applying energy dependent classifier output cuts the electron contamination is estimated at about ~$5\%$
				\item This is calculated by reweighing the spectrum of dedicated proton and electron simulations to realistic spectra for previous experiments
				\item We also estimate the live time for the reweighed simulations to create an event rate for each CR species and then find the residual contamination in the proton signal
				\item The estimated electron contamination after applying the energy dependent proton classifier ranges from less than $1\%$ to down to $1 \times 10^{-2}\%$ of residual contamination
				\item We also find the contamination for the different scanning efficiencies 
				\item The highest contamination is at the highest energies where the distinguishing power of the LAT between hadronic and leptonic showers is greatly reduces due the LAT's small size and large size of said particle showers
				\item This is not really a problem because even with out the proton classifier the electron contamination is still very small
			\end{myEnumerate}
		\end{myEnumerate}
		\item Spectral Reconstruction
		\begin{myEnumerate}
			\item There are two methods we use to reconstruct the incident proton cosmic-ray spectrum: Unfolding and Forward Folding
			\begin{myEnumerate}
				\item The reason for a folding method is because the reconstructed energy does not completely recreate the true energy as is done with the electron and gamma-ray events
				\item We must use a response matrix to determine the event rates in true energy from reconstructed energy
				\item The inversion problem means we have to use one of these methods in order to find the true spectrum
				\item The two methods are analogous and provide a good check to make sure the folding process is working and everything is understood
			\end{myEnumerate}
			\item Unfolding
			\begin{myEnumerate}
				\item Using TUnfold an unfolding package based in ROOT
				\item We use a Thikonov regularization (based on curvature) and minimize the regularization term based on a tau scan (Thikonov is a standard unfolding regularization method)
				\item Unfolding is nice because it does not require a model for the incident particle spectrum and therefore makes no assumption of the about the incident proton spectrum
				\item Using a response matrix determine from proton simulations with each of the event selection cuts (a response matrix for each proton classifier cut as well)
				\item The true energy binning is determined from the energy resolution from the proton simulation with all cuts (except for the proton classier cuts) applied
				\item The reconstructed energy binning is just set to be larger than the binning of the true energy binning (under sampled case)
				\item The minimum true energy bin is determine above
				\item The maximum true energy bin is determine the number of events to ensure low statistical uncertainties
				\item Data file taken from flight data with each event selection
				\item The output of the unfolding is the event rate in true energy which is used to compute the differential flux spectrum
			\end{myEnumerate}
			\item Forward Folding
			\begin{myEnumerate}
				\item Forward folding assumes an input spectral model, takes the acceptances, response matrix, and contamination and creates an event rate in true energy by minimizing differences between reconstructed flight event rates and data flight rate
				\item Can try different spectral models including power-law and broken power-law
				\item Determine which model has the lowest $\chi^2_{red}$ and best spectral model
				\item Unfortunately, have to assume a spectral model
				\item The true energy binning is determined from the energy resolution from the proton simulation with all cuts (except for the proton classier cuts) applied
				\item The reconstructed energy binning is just set to be larger than the binning of the true energy binning (under sampled case)
				\item The minimum true energy bin is determine above
				\item The maximum true energy bin is determine the number of events to ensure low statistical uncertainties
				\item Data file taken from flight data with each event selection
				\item The output of the unfolding is the event rate in true energy which is used to compute the differential flux spectrum
			\end{myEnumerate}
			\item Once the event rate is determined from either unfolding and forward folding, we subtract the estimated contamination from the event rate in true energy, divide the by the acceptances, and divide each bin by the bin-width in true energy to get the differential energy dependent flux spectrum
			\item This is done for the optimized event selection and each scanning efficiency
			\item The scanning efficiencies are used to probe systematic errors, which is discussed next
		\end{myEnumerate}
		\item Systematic Uncertainties
		\begin{myEnumerate}
			\item Acceptance Uncertainties
			\begin{myEnumerate}
				\item Systematic Uncertainties in the acceptance are determined from the scanning efficiencies of the proton classifier
				\item Each proton scanning efficiency will create a different acceptance, different event rate, and different contamination
				\item  If proton classifier and simulations well represent the data then each spectrum will give similar results
				\item This is very similar to the bootstrapping method in determining the systematic Uncertainties for gamma-ray data for the LAT
				\item These Uncertainties are energy dependent and increase as incident energy increases
				\item They are estimated to be $X\%$ at $\sim$100 GeV and $X\%$ at $\sim$1 TeV \textcolor{red}{I need to figure out how large these actually are compared to the optimized selection}
			\end{myEnumerate}
			\item Contamination Uncertainties
			\begin{myEnumerate}
				\item Systematic uncertainties associated with the contaminations are expected to be low
				\item Since the estimated contaminations are low for both the electron and heavy ion contamination we do not expect any systematic Uncertainties to significantly effect the resultant spectrum
				\item Heavy Ion Uncertainties would have the largest effect since the heavy ion Geant4 simulations are pretty poor but the estimated contamination is still below $1\%$ at the lowest energy
				\item It would be reasonable to assume the heavy ion contamination at lowest energy is $\sim 1\%$ and there fore negligible compared to other systematic Uncertainties
				\item Electron contaminations are handled in a similar way to acceptance Uncertainties with the different constant scanning efficiencies from the proton classifier
			\end{myEnumerate}
			\item Geant4 Simulation Uncertainties
			\begin{myEnumerate}
				\item Geant4 has potential uncertainties in the simulation of protons
				\item We can determine the quality of simulations by examining accelerator experiments with much more control over their particle sources like CALICE and LHC
				\begin{myEnumerate}
					\item Geant4 tends to underestimate both transverse and longitudinal shower development by up to to $\sim 15\%$
					\item Geant4 uncertainties on relative energy resolutions for protons are considered small
					\item Absolute energy deposition is under $10\%$
				\end{myEnumerate}
				\item Here is are the physics lists that GR's version of Geant4 use specifically for proton simulations \textcolor{red}{This is something I need to look up really soon such that we can have a better understanding what what we are dealing with.  I am sure it is just sitting somewhere in GR but I still have to look for it... sigh...}
				\begin{myEnumerate}
					\item  Physics List stuff
				\end{myEnumerate}
				\item We can also examine Pass 8 reconstructed beamtest data from 2006.  
				\begin{myEnumerate}
					\item Original beamtest data was taken in 2006 with a spare flight towers called the ``Calibration Unit''
					\item We can compare this beamtest data to simulations and get an idea quality of simulations when looking at energy deposition, shower size, and shower width
					\item We can also use the BT data to confirm what other accelerator experiments have observed with Geant4 uncertainties
				\end{myEnumerate}
			\end{myEnumerate}
			\item Energy Measurement Uncertainties
			\begin{myEnumerate}
				\item To estimate systematic Uncertainties associated with the energy measurement, specifically with energy resolution and abolsute energy scale, there are two ways of doing so
				\item Geomagnetic Ridigity cutoff
				\begin{myEnumerate}
					\item This would entail looking at the geomagnetic rigidity cut off and observing proton event rates as function of McIllwain L parameter
					\item It would involve using diagnostic data and ray tracing code in order to distinguish trapped protons from cosmic-ray protons
					\item GRC provides a feature at specific energies that can be used to estimate the absolute energy scale
					\item UNFORTUNETLY the tracer code is highly dependent on energy resolution as we have seen the energy resolution is pretty poor for protons
					\item This method seems very involved and has limited probing capability and predominantly works at lower energies
				\end{myEnumerate}
				\item Beamtest Data and Simulation
				\begin{myEnumerate}
					\item Original beamtest data was taken in 2006 with a spare flight towers called the ``Calibration Unit''
					\item Several configurations were taken of proton data:

  \begin{tabular}{| c | c | c | c |}
    \hline
    Particle & Energy (GeV) & Incident Angle (degrees) & Number of Events \\ \hline
    Proton & 20 & 0 & 100211 \\ \hline
    Proton & 100 & 0 & 37183 \\ \hline
    Proton & 100 & 45 & 28128 \\ \hline
    Proton & 100 & 90 & 24784 \\ \hline
    Proton & 150 & 0  & 8958\\ \hline
  \end{tabular}
  					\item While not ideal, we can use the data/mc agreement to estimate systematic errors associated with absolute energy scale and energy resolution
  					\item Preliminary study suggests that there is generally good agreement between Pass 8 Beamtest data and simulations
  					\item Absolute energy scale for protons agrees within $\pm 5\%$ and relative energy resolution is under $\pm 10\%$
  					\item Absolute energy scale is more important that relative energy resolution, since the absolute energy scale is a parameter that determines the absolute flux
				\end{myEnumerate}
			\end{myEnumerate}
			\item Spectral Reconstruction Uncertainties
			\begin{myEnumerate}
				\item We can explore potential systematic errors in the energy redistribution of event by trying two different methods
				\item Unfolding and forward folding are two methods to solve the inverse problem
				\item They should produce the same results and give a nice probe into potential systematics
				\item \textcolor{red}{I would not expect this to be a large effect, smaller than acceptance or energy measurement but it is something else to check.  I am thinking of a possible way of estimating the effect but it might also be easier to run unfolding and forward folding together and combine the results since we already have code for forward folding established.  If it turns out to be small it is still a nice check to make sure the unfolding processes is working correctly, which it totally should be.}
			\end{myEnumerate}
		\end{myEnumerate}
	\end{myEnumerate}
\section{Results and Discussion}
	\begin{myEnumerate}
		\item 
	\end{myEnumerate}

\end{document}