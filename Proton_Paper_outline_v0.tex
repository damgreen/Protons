\documentclass{article} 
\usepackage[margin=1.0in]{geometry}
\usepackage{outlines}
\usepackage{lineno}
\usepackage{color}

\linenumbers
\usepackage{enumitem}
\setlistdepth{8}

\newlist{myEnumerate}{enumerate}{9}
\setlist[myEnumerate,1]{label=(\Roman*)}
\setlist[myEnumerate,2]{label=(\Alph*)}
\setlist[myEnumerate,3]{label=(\roman*)}
\setlist[myEnumerate,4]{label=(\alph*)}
\setlist[myEnumerate,5]{label=(\arabic*)}
\setlist[myEnumerate,6]{label=(\Roman*)}
\setlist[myEnumerate,7]{label=(\Alph*)}
\setlist[myEnumerate,8]{label=(\roman*)}

\begin{document} 
\author{David M. Green}
\title{Fermi-LAT Measurement of Cosmic-ray Proton Spectrum \\ Paper Outline - Version 0}
\date{\today}

\maketitle

\begin{abstract}
The Pass 8 gamma-ray simulation and reconstruction package for the Large Area Telescope (LAT) on the Fermi Gamma-ray Space Telescope has allowed for the development of a new cosmic-ray proton analysis. 
Using the Pass 8 direction and energy reconstruction, we create a new proton event selection. 
This event selection has an acceptance of 1 m$^2$ sr over the incident proton energy range from 50 GeV to over 8 TeV and when applied to over 7 years of LAT observations provides over 700 million events for a spectral measurement. 
The systematic errors in the acceptance and energy reconstruction require careful study and will contribute significantly to the spectral measurement. 
The event selection and spectral measurement of the Pass 8 proton analysis opens the door to additional proton analyses with the LAT, such as the evaluation of proton anisotropy. 
We present a detailed study on the measurement of the cosmic-ray proton spectrum with Pass 8 data for the Fermi LAT.
\end{abstract}

\section{Introduction}
	\begin{myEnumerate}
		\item Describe overview of LAT
		\item Event selection for high quality proton sample
		\item Energy reconstruction, biases, energy resolution, and limitations
		\item Describe out instrument response: acceptance and contamination
		\item Describe the methods used for spectral reconstruction: unfolding and forward folding using response matrix derived from MCs
		\item Describe evaluation of systematic uncertainties
		\begin{myEnumerate}
			\item Due to event selection: acceptance and contamination
			\item Energy measurement: absolute energy scale and energy resolution
			\item From hadronic model of Geant4 simulations
			\item Spectral reconstruction: comparing unfolding and forward folding methods
		\end{myEnumerate}
		\item Finally discuss observations and features of measured spectral, including possible spectral break and agreement with recent results \textcolor{red}{(definitely need to but this in context with other measurements since while energy resolution is poor and systematics less precises than AMS-02 we can extend the energy further into the region of balloon-borne detectors which have never been done before and makes a quantitative connection between two different observation environments)}
	\end{myEnumerate}

\section{Event Analysis}
	\begin{myEnumerate}
		\item Overview
			\begin{myEnumerate}
				\item Description of the LAT
				\item 4$\times$4 array of towers which measure direction and energy of incoming cosmic-ray
				\item Each tower is composed of TKR and CAL
				\item TKR information
				\begin{myEnumerate}
					\item Each TKR module is 18 x-y planes of silicon-strip detectors with tungsten converter foil
					\item Total of 1.5 $X_0$ at normal incidence (should convert this to nuclear interaction length)
					\item X-Y nature and depth of TKR allows for determination of initial direction of cosmic-ray
					\item Additionally able to measure the time over threshold of CR
					\item ToT allows for measurement of signal $\propto$ Z$^2$
				\end{myEnumerate}
				\item CAL information
				\begin{myEnumerate}
					\item CAL is homogeneous electromagnetic calorimeter
					\item Each CAL module is 96 CsI(Tl) crystals in an hodoscopic array in 8 layers.
					\item The hodoscopic nature of the CAL allows for measuring the shape and evolution of each particle shower which can be used with a profile fitter to determine the incident energy of the cosmic-ray
					\item At normal incidence the CAL is 0.5 $\lambda_i$ lengths deep but at horizontal incidence is it 1.5 $\lambda_i$ deep
				\end{myEnumerate}
				\item Anti-coincidence detector (ACD) surrounds the 4$\times$4 tower array
				\item ACD information
				\begin{myEnumerate}
					\item 89 segmented covering 5 sides of the tower array
					\item Each tile independently measures deposited energy from CR
					\item Deposited energy $\propto$ Z$^2$ 
				\end{myEnumerate}
				\item LAT was not designed for accurate measurement of hadronic showers
				\begin{myEnumerate}
					\item Very shallow homogeneous calorimeter not idea for fully capturing energy hadronic shower profile
					\item Compare to CREAM and/or AMS-02 
					\item Unable to measure energy on an event by event basis, need to focus on a statistical ensemble approach with high event rate
					\item Therefore need to be aware of limitation of energy measurement and associated systematic uncertainties
				\end{myEnumerate}
			\end{myEnumerate}
		\item Monte Carlo simulations
		\item Event Selection
		\item Energy reconstruction
	\end{myEnumerate}

\section{Spectral Analysis}
	\begin{myEnumerate}
		\item Instrument Acceptance
		\item Residual Contamination
		\item Spectral Reconstruction
		\item Systematic Uncertainties
	\end{myEnumerate}
\section{Results and Discussion}


\end{document}