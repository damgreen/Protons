\documentclass{article} 
\usepackage[margin=1.0in]{geometry}
\usepackage{outlines}
\usepackage{lineno}
\usepackage{color}

\linenumbers
\usepackage{enumitem}
\setlistdepth{8}

\newlist{myEnumerate}{enumerate}{9}
\setlist[myEnumerate,1]{label=(\Roman*)}
\setlist[myEnumerate,2]{label=(\Alph*)}
\setlist[myEnumerate,3]{label=(\roman*)}
\setlist[myEnumerate,4]{label=(\alph*)}
\setlist[myEnumerate,5]{label=(\arabic*)}
\setlist[myEnumerate,6]{label=(\Roman*)}
\setlist[myEnumerate,7]{label=(\Alph*)}
\setlist[myEnumerate,8]{label=(\roman*)}

\begin{document} 
\author{David M. Green}
\title{Fermi-LAT Measurement of Cosmic-ray Proton Spectrum \\ Paper Outline - Version 0}
\date{\today}

\maketitle

\begin{abstract}
The Pass 8 gamma-ray simulation and reconstruction package for the Large Area Telescope (LAT) on the Fermi Gamma-ray Space Telescope has allowed for the development of a new cosmic-ray proton analysis. 
Using the Pass 8 direction and energy reconstruction, we create a new proton event selection. 
This event selection has an acceptance of 1 m$^2$ sr over the incident proton energy range from 50 GeV to over 8 TeV and when applied to over 7 years of LAT observations provides over 700 million events for a spectral measurement. 
The systematic errors in the acceptance and energy reconstruction require careful study and will contribute significantly to the spectral measurement. 
The event selection and spectral measurement of the Pass 8 proton analysis opens the door to additional proton analyses with the LAT, such as the evaluation of proton anisotropy. 
We present a detailed study on the measurement of the cosmic-ray proton spectrum with Pass 8 data for the Fermi LAT.
\end{abstract}

\section{Introduction}
	\begin{myEnumerate}
		\item Describe overview of LAT
		\begin{myEnumerate}
			\item Launch date to give early context of how much data there is available
			\item Orbital parameters to show what kind of space environment we have to deal with
			\item Development of Pass 8, short list of improvements and how this enables use to make a new proton analysis with the LAT
		\end{myEnumerate}
		\item Discuss recent developments of the CR proton spectrum from instruments
		\begin{myEnumerate}
			\item AMS-02 observes break in spectrum at 300 GeV
			\item Potentially resolves discrepancy between satellite measurements in 100s GeV energy and balloon-borne measurements 
			\item But.... AMS-02 only goes to 1.8 TeV, statistics limited due to small acceptance and X years of flight
			\item Gap left between 1.8 TeV of AMS-02 and 3 TeV of CREAM
		\end{myEnumerate}
		\item Goals of this analysis
		\begin{myEnumerate}
			\item Measure the cosmic-ray proton spectrum from 50ish GeV to several TeV
			\item Fermi LAT in unique position to measure spectrum spanning between satellite measurements and balloon borne measurements
			\item Also able to confirm spectral break as currently only seen by AMS-02 and possibly by Pamela
			\item Create a new data set of cosmic-ray protons for future analysis \textcolor{red}{(I'm not sure we really need this in the paper but might be nice to mention)}
		\end{myEnumerate}
		\item Event selection for high quality proton sample
		\item Energy reconstruction, biases, energy resolution, and limitations
		\item Describe out instrument response: acceptance and contamination
		\item Describe the methods used for spectral reconstruction: unfolding and forward folding using response matrix derived from MCs
		\item Describe evaluation of systematic uncertainties
		\begin{myEnumerate}
			\item Due to event selection: acceptance and contamination
			\item Energy measurement: absolute energy scale and energy resolution
			\item From hadronic model of Geant4 simulations
			\item Spectral reconstruction: comparing unfolding and forward folding methods
		\end{myEnumerate}
		\item Finally discuss observations and features of measured spectral, including possible spectral break and agreement with recent results \textcolor{red}{(definitely need to but this in context with other measurements since while energy resolution is poor and systematics less precises than AMS-02 we can extend the energy further into the region of balloon-borne detectors which have never been done before and makes a quantitative connection between two different observation environments)}
	\end{myEnumerate}

\section{Event Analysis}
	\begin{myEnumerate}
		\item Overview
			\begin{myEnumerate}
				\item Description of the LAT
				\item 4$\times$4 array of towers which measure direction and energy of incoming cosmic-ray
				\item Each tower is composed of TKR and CAL
				\item TKR information
				\begin{myEnumerate}
					\item Each TKR module is 18 x-y planes of silicon-strip detectors with tungsten converter foil
					\item Total of 1.5 $X_0$ at normal incidence (should convert this to nuclear interaction length)
					\item X-Y nature and depth of TKR allows for determination of initial direction of cosmic-ray
					\item Additionally able to measure the time over threshold of CR
					\item ToT allows for measurement of signal $\propto$ Z$^2$
					\item The last 4(?) Tungsten converter foils are thicker than the previous layers to ensure gamma-rays convert within TKR
				\end{myEnumerate}
				\item CAL information
				\begin{myEnumerate}
					\item CAL is homogeneous electromagnetic calorimeter
					\item Each CAL module is 96 CsI(Tl) crystals in an hodoscopic array in 8 layers.
					\item The hodoscopic nature of the CAL allows for measuring the shape and evolution of each particle shower which can be used with a profile fitter to determine the incident energy of the cosmic-ray
					\item additional the imagine capability of the CAL allows for the measurement of the direction of the incident CR
					\item At normal incidence the CAL is 0.5 $\lambda_i$ lengths deep but at horizontal incidence is it 1.5 $\lambda_i$ deep
				\end{myEnumerate}
				\item Anti-coincidence detector (ACD) surrounds the 4$\times$4 tower array
				\item ACD information
				\begin{myEnumerate}
					\item 89 segmented covering 5 sides of the tower array
					\item Each tile independently measures deposited energy from CR
					\item Deposited energy $\propto$ Z$^2$ 
				\end{myEnumerate}
				\item Description of the LAT triggers and filters and point towards the paper with more information
				\item LAT was not designed for accurate measurement of hadronic showers
				\begin{myEnumerate}
					\item Very shallow homogeneous calorimeter not idea for fully capturing energy hadronic shower profile
					\item Compare to CREAM and/or AMS-02 
					\item Unable to measure energy on an event by event basis, need to focus on a statistical ensemble approach with high event rate
					\item Therefore need to be aware of limitation of energy measurement and associated systematic uncertainties
				\end{myEnumerate}
			\end{myEnumerate}
		\item Pass 8 Event Reconstruction 
			\begin{myEnumerate}
				\item \textcolor{red}{I'm not 100\% sure of the depth of this section but seeing as though we are using Pass 8 and that was a somewhat critical step into enabling this analysis's possibility I think having a dedicated section in the Event analysis chapter might make sense.  If we put it anywhere it should be rather early before the simulations and after the describing the instrument}
				\item Pass 8 is the new event reconstruction and simulation software developed by the Fermi LAT collaboration that drastically improves LAT's performance
				\item New event classification using boosted decision trees in TMVA
				\begin{myEnumerate}
					\item Several new Pass 8 variables have been created to determine the quality of the direction, energy, and gamma-ray quality
				\end{myEnumerate}
				\item More variables gives better separation between hadronic and leptonic showers in TMVA
				\item Improved profile to fitting to particle showers improves energy measurement
				\begin{myEnumerate}
					\item The New Full Profile fitter is able to extend the longitudinal profile of the shower outside the CAL therefore estimating the amount of energy leakage for high energy events, $>$ 100 GeV
					\item Two energies derived from new full profile fitter, one for TKR directions and one for CAL directions
				\end{myEnumerate}
				\item New tree based TKR reconstruction allows for direction reconstruction at higher angles and larger energies
				\item New ACD reconstruction provides better particle identification, lowering the contamination of proton sample
				\item \textcolor{red}{Is there something else I am missing from Pass 8?  There is no Pass 8 paper to reference this so I am not sure how in depth I should go into this discussion.}
			\end{myEnumerate}
		\item Monte Carlo simulations
			\begin{myEnumerate}
				\item Need to stress the importance of the simulations since this is how we derive all of our instrument response functions 
				\item Also use simulations for the development of TMVA selection to remove contamination for other CRs
				\item Simulations based on Geant4
				\item LAT instrument and spacecraft are fully simulated within Geant4
				\item Particles with distributions of energies, directions, and charges are generated and propagated with realistic physics models for interactions with the simulated LAT which create raw data
				\item Raw simulated data is processed through the same Pass 8 reconstruction software as flight data
				\item We preform extensive comparison between simulated data and flight data to ensure results from MC analyses can be reliably applied flight data
				\item Three types of simulations are used this analysis:
				\item Proton simulation
				\begin{myEnumerate}
					\item Simulation run from 4 GeV to 20 TeV
					\item Cover 4$\pi$ sr 
					\item Created with an $dN/dE \propto E^{-1.5}$ spectral index
					\item Original purpose to study Pass 8 CR rejection for studying extragalactic background light 
					\item This produces a simulation event sample of over X million events
				\end{myEnumerate}
				\item Electron simulation
				\begin{myEnumerate}
					\item  10 GeV to 10 TeV
					\item Cover 2$\pi$ sr (the top half of the instrument)
					\item Created with a $dN/dE \propto E^{-1.0}$ spectral index
					\item Original purpose of studying instrument response for cosmic-ray electron analysis
					\item This produces a simulation event sample of over X million events
				\end{myEnumerate}
				\item Background simulation 
				\begin{myEnumerate}
					\item The background simulation was created to accurately simulate the cosmic-ray environment of the LAT during space flight
					\item It contains CR particle from Z = 1 to Z = 26, electrons, positrons, neutrons, and Earth albedo gamma-rays
					\item All particles are simulated with realistic fluxes using results from recent CR experiments
					\item The background simulation used in this analysis simulates about 8 days worth of livetime 
					\item Protons range: 4 GeV - 10 TeV and 4$\pi$ sr
					\item Electrons/Positrons range: 4 GeV - 10 TeV and 4$\pi$ sr
					\item Helium range: 4 GeV - 20 TeV and 4$\pi$ sr
					\item Heavier CR range: 2 GeV/amu - 50 GeV/amu and 4$\pi$ sr
					\item Fluxes are taken to be near solar minimum
				\end{myEnumerate}
				\item All simulations are produces with an additional setting called overlay events
				\begin{myEnumerate}
					\item Overlay events are created from diagnostic events from flight data and signal is added on top of the simulated data
					\item This is mimic the effect of having two events simultaneous enter the LAT (for such high events rates at lower energies is a reasonable assumption)
					\item Pass 8 has many new algorithms to handle and reduce the effect of two simultaneous events interacting with the LAT
				\end{myEnumerate}

			\end{myEnumerate}
		\item Event Selection
			\begin{myEnumerate}
				\item Minimum Quality Cuts
				\begin{myEnumerate}
					\item We want a selection of hight quality protons for the spectral analysis
					\item TkrNumTracks $>$ 0: Require an event to have at least one track
					\item WP8CTPSFTail $>$ 0.5: WP8CTPSFTail is a new Pass 8 TMVA variable which determines the quality of direction reconstruction, this ensures the track is well reconstructed
					\item CalEnergyRaw $>$ 20 GeV: This utilizes the high pass filter on the LAT.  Any event with a deposited energy greater than 20 GeV is downloaded from the LAT.  This means for events with CalEnergyRaw $>$ 20 GeV is not effected by and lower energy filters which are difficult to understand for protons
					\item CalTrackAngle $<$ 0.3: Ensure the difference between the CAL and TKR directions is small.
					\item CalNewCfpSat: Ensure that the variables resulting new full profile fitter are not saturated
					\item Tkr1LengthInCal $>$ 200.0: Want a long path length through the calorimeter and does not fall within gaps of CAL.  More active material will help ensure more deposited energy and a better reconstructed energy
					\item log10(TkrTree1ThickRLnNodes) $<$ 1.0: To make sure events convert within the beginning of CAL and not last few layers of TKR (don't want to lose energy to TKR) we ensure the last few layers of the TKR do not have too many events  This also helps with backslash and again losing energy back into the TKR
					\item These cuts ensure the direction and energy are well reconstructed, still need remove contamination source from helium + heavier CRs and electrons
				\end{myEnumerate}
				\item Helium and Heavier Ion Cut
				\item Proton Classifier
			\end{myEnumerate}
		\item Energy reconstruction
			\begin{myEnumerate}
				\item 
			\end{myEnumerate}
	\end{myEnumerate}

\section{Spectral Analysis}
	\begin{myEnumerate}
		\item Instrument Acceptance
		\item Residual Contamination
		\item Spectral Reconstruction
		\item Systematic Uncertainties
	\end{myEnumerate}
\section{Results and Discussion}


\end{document}