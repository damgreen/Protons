\documentclass{article}
\usepackage{helvet}
\usepackage[margin=1.0in]{geometry}
\usepackage{graphicx}
\usepackage{lineno}
\usepackage{color}
\usepackage{amsmath,accents}
\usepackage{mathrsfs}

\renewcommand*{\familydefault}{\sfdefault}
\linenumbers


\title{Fermi-LAT Measurement of Cosmic-ray Proton Spectrum \\ Draft 0}
\author{David M. Green}
\date{\today}

% \setcounter{tocdepth}{0}

\begin{document}

\maketitle

\begin{abstract}
The Pass 8 gamma-ray simulation and reconstruction package for the Large Area Telescope (LAT) on the Fermi Gamma-ray Space Telescope has allowed for the development of a new cosmic-ray proton analysis. 
Using the Pass 8 direction and energy reconstruction, we create a new proton event selection. 
This event selection has an acceptance of 200 cm$^2$ sr over the incident proton energy range from 50 GeV to over 8 TeV and when applied to over 7 years of LAT observations provides over 700 million events for a spectral measurement. 
The systematic errors in the acceptance and energy reconstruction require careful study and will contribute significantly to the spectral measurement.
The event selection and spectral measurement of the Pass 8 proton analysis opens the door to additional proton analyses with the LAT, such as the evaluation of proton anisotropy. 
We present a detailed study on the measurement of the cosmic-ray proton spectrum with Pass 8 data for the Fermi LAT.
\end{abstract}

\input{section01}
\section{Event Analysis}

\subsection{Overview}

\subsection{Pass 8 Event Reconstruction}

\subsection{Monte Carlo Simulations}

\subsection{Event Selection}

\subsection{Energy Measurement}
\section{Spectral Analysis}

\subsection{Instrument Acceptance}

\subsection{Residual Contamination}

\subsection{Spectral Reconstruction}

\subsection{Systematic Uncertainties}
\input{section04}

% \newpage
\bibliographystyle{abbrv}%Choose a bibliograhpic style

\bibliography{proton_bib}

\end{document}
